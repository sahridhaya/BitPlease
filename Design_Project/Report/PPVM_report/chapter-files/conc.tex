\chapter{CONCLUSIONS}\label{conc}
\thispagestyle{empty}

The prototype of the PPVM was successfully completed. Initially the accuracy of detection was not excellent. The accuracy was improved by increasing the number of epochs while training the model. It is also expected that with further usage of this machine, it will become more and more accurate, as mode samples will be added to the dataset.

Through this project, the following concepts were learned
\begin{itemize}
	\item Basics of TensorFlow and Machine Learning
	\item Basics of Raspberry Pi
	\item Basics of OpenCV
	\item Design and function of a vending machine
\end{itemize}

Many issues were faced while working on Raspberry Pi. It was difficult to install TensorFlow and OpenCV in Pi and took many hours. The HDMI port of Pi stopped working after a while. Further work on Pi was done by enabling remote access using
a laptop. Even this method was not perfect and it failed to connect with three different laptops. While dealing with each problem, new concepts were learned about programming, designing, and networking. 

The implementation of this machine is expected to mitigate the emission of CO$_2$, as mentioned in Chapter 3. For this, a stable and durable model needs to be constructed. More studies need to be conducted for perfecting and implementing this machine.