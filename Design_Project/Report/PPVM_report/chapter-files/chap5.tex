\chapter{DISCUSSIONS}\label{chap5}
\thispagestyle{empty}

During the initial phase of the project, the problem statement had to be analyzed through data collection and analysis. Thus, small scale surveys were conducted in GECBH using standard methods. After a while, a better way to conduct these surveys was found and hence ODK was adopted as the prime tool.

While discussing the software part of the machine, the initial thought was to go with image processing. But this method was proven to be more time-consuming. A better way to tackle the issue was pattern recognition. It was soon understood that there was abundant documentation available on the internet regarding pattern recognition and image classification. Thus, TensorFlow - a utility by Google - was chosen for this task.

The body of the machine needed to be sturdy and strong enough to withstand the components and the rotating servo. For the prototype, it was decided that a combination of foam board and  acrylic sheet would be the best. The bigger parts would be joined together using clamps and the smaller ones with glue.

To make the machine autonomous, a microprocessor needed to be attached to the machine. This processor should also be able to act as a microcontroller and the best candidate for this was Raspberry Pi.
