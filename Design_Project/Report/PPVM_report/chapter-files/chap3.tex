\chapter{DESIGN/DATA COLLECTION}\label{chap3}
\thispagestyle{empty}

An introduction to this chapter has to be provided here. This chapter should provide the design that you made by numerical, analytical, simulation, or observation based methods. Each section below this should focus on these aspects of your design. 
\section{A Section Here}

A figure can be inserted as follows. Fig.~\ref{figureone} is in this section and so on...... First observe the figure number should contain the chapter number as the first digit and the second digit or number as the position of the figure in the current chapter.  

\begin{figure}[h!]
\centering
\includegraphics[width=0.4\linewidth]{./picture-files/gecb1}
\caption[A sample figure]{A sample figure inserted in a chapter}
\label{figureone}
\end{figure}

See how the figure in this chapter is used in another chapter to refer to it by its number and page number. Check Chapter~\ref{chap4}, page no.~\pageref{chap4}.

\section{Second Section}

Here in this section you can give a proper name and explain about it. 

We can see a sample table, Table~\ref{tableone}, in page no.~\pageref{tableone} referred in this section. Any floating objects like this can be referred without actually counting the page where it comes in the document. Just say what to be done, the rest is up to \LaTeX. % will do the rest.
\begin{table}[h!]\centering \caption{Expenses of Rakhul}\label{tab-exp}
\begin{tabular}{|l|r|r|r|}
\hline 
Item	&	Rate	&	Qty.	&	Amount	\\ \hline
Rice	&	34	&	5	&	170	\\ \hline
Sugar	&	32	&	1	&	32	\\ \hline
Salt	&	15	&	1	&	15	\\ \hline
Chilli	&	150	&	0.25	&	37.5	\\ \hline
	&		&	Total	&	254.5	\\ 
	\hline
\end{tabular}

\end{table}

\begin{table}[h!]\centering \caption{Modifications in a table design}
\begin{tabular}{||c||c||c||c||c||}
\hline Rakhul & Vrinda & Raveendran & Krishna & Anu \\ 
\hline  &  &  &  &  \\ 
\hline  &  &  &  &  \\ 
\hline  &  & Anna & Bhaskar & Nizam \\ 
\hline 
\end{tabular} 
\end{table}


In Section~\ref{e-proc}, page no.~\pageref{e-proc}, the different modes and different practices in e-procurement has been discussed. The research in e-procurement actually discusses the success stories of e-procurement.
\section{Equation referred here}

Any equation in the report can be referred anywhere like this. Eqn.~\eqref{dist}, page no.~\pageref{dist}  is a sample equation that says about the displacement of an object travelling with specific parameters. 

\section{Summary}

Provide a paragraph to summarise every chapter.